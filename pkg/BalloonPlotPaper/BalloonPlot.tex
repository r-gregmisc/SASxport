\documentclass[a4paper]{report}
\usepackage{Rnews}
\usepackage[round]{natbib}
\bibliographystyle{abbrvnat}

\begin{document}

\begin{article}
\title{Balloon Plot}
\subtitle{Graphical tool for displaying tabular data}
\author{by Nitin Jain and Gregory R. Warnes}

\maketitle

\section*{Introduction}

Numeric data is often summarized using rectangular tables. While
these tables allow presentation of all of the relevant data, they do
not lend themselves to rapid discovery of important patterns. The
primary difficulty is that the visual impact of numeric values is
not proportional to the scale of the numbers represented.

We have developed a new graphical tool, the ``balloonplot'',
which augments the numeric values in tables with colored circles
with area proportional to the size of the corresponding table
entry. This visually highlights the prominent features of
data, while preserving the details conveyed by the numeric values
themselves.

In this article, we describe the balloonplot, as implemented by the
\code{balloonplot} function in the \code{gplots} package, and
describe the features of our implementation.  We then provide an
example using the ``Titanic'' passenger survival data.  We conclude
with some observations on the balloonplot relative to the
previously developed ``mosaic plot''.

\section*{Function description}

The \code{balloonplot} function accepts a table (to be displayed as
found) or lists of vectors for x (column category), y (row category)
and z (data value) from which a table will be constructed.

The \code{balloonplot} function creates a graphical table
where each cell displays the appropriate numeric value plus a
colored circle whose size reflects the relative magnitude of the
corresponding component. The
\emph{area} of each circle is proportional to the frequency of
data. (The circles are scaled so that the circle for largest value
fills the available space in the cell.)


As a consequence, the largest values in the table are
``spotlighted'' by the biggest circles, while smaller values are
displayed with smaller circles.  Of course, circles can only have
positive radius, so the radius of circles for cells with negative
values are set to zero.  (A warning is issued when this
``truncation'' occurs.)

Of course, when labels are present on the table or provided to the
function, the graphical table is appropriately labeled.  In
addition, options are provided to allow control of various visual features
of the plot:

\begin{itemize}
  \item rotation of the row and column headers
  \item balloon color and shape (globally or individually)
  \item number of displayed digits
  \item display of entries with zero values
  \item display of marginal totals
  \item display of cumulative histograms
  \item x- and y-axes group sorting
  \item formatting of row and column labels
  \item traditional graphics parameters (title,
    background, etc.)
\end{itemize}

\section*{Example using the \code{Titanic} data set}

For illustration purposes, we use the \code{Titanic} data set from
the \code{datasets} package.  \code{Titanic} provides survival status
for passengers on the tragic maiden voyage of the ocean liner
``Titanic'', summarized according to economic status (class), sex, and
age.

Typically, the number of surviving passengers are shown in a tabular
form, such as shown in Figure~\ref{figure:Figure1}.  (This was created
by calling balloonplot with the balloon color set to match the
background color and most options disabled.)  Note that one must
actively focus on the individual cell values in order to see any
pattern in the data.

\begin{figure}
\includegraphics[width=\textwidth]{Figure1.pdf}
\caption{\label{figure:Figure1}
Tabular representation of survived population by gender and age}
\end{figure}


Now, we redraw the table with light-blue circles (`balloons')
superimposed over the numerical values (Figure~\ref{figure:Figure2}).
This is accomplished using the code:

{\small
\begin{verbatim}
library(gplots)
data(Titanic)

# Convert to 1 entry per row format
dframe <- as.data.frame(Titanic) 

# Select only surviving passengers
survived <- dframe[dframe$Survived=="Yes",]
attach(survived)

balloonplot(x=Class,
            y=list(Age, Sex),
            z=Freq,
            sort=TRUE,
            show.zeros=TRUE,
            cum.margins=FALSE,
            main=
            "BalloonPlot : Surviving passengers"
            )

title(main=list("Circle area is proportional to\
 number of passengers",
           cex=0.9),
      line=0.5)

detach(survived)
\end{verbatim}
}
%$

\begin{figure}
\includegraphics[width=\textwidth]{Figure2.pdf}
\caption{\label{figure:Figure2}
Balloon plot of surviving individuals by class, gender and age }
\end{figure}

With the addition of the blue ``spotlights'', whose area is
proportional to the magnitude of the data value, it is easy to see
that only adult females and adult male crew members survived in
large numbers.  Also note the addition of row and column marginal
totals.

Of course, the number of surviving passengers is only half of the
story.  We could create a similar plot showing the number of
passengers who did not survive.  Alternatively, we can simply add
survival status as another variable to the display, setting the
color of the circles to green for passengers who survived, and
magenta for those who did not (Figure 3).  This conveys considerably
more information than Figures 1 and 2 without substantial loss of
clarity. The large magenta circles make it clear that most
passengers did not survive.


\begin{figure}
\includegraphics[width=\textwidth]{Figure3.pdf}
\caption{\label{figure:Figure3}
  Balloon plot of Titanic passengers by gender, age and class. Green
  circles represent passengers who survived and magenta circles
  represent the passengers who did not survive.}
\end{figure}


To further improve the display, we add a visual
representation of the row and column sums (Figure
\ref{figure:Figure4}).  This is accomplished using light grey bars
behind the row and column headers.  The length of each bar is
proportional to the corresponding sum, allowing rapid visual
ascertainment of their relative sizes.  We have also added
appropriately colored markers adjacent to the headers under
``Survived'' to emphasize the meaning of each color.

{
\small
\begin{verbatim}
attach(dframe)
colors <- ifelse( Survived=="Yes", "green", 
                  "magenta")

balloonplot(x=Class,
            y=list(Survived, Age, Sex),
            z=Freq,
            sort=FALSE,
            dotcol=colors,
            show.zeros=TRUE,
            main="BalloonPlot : Passenger Class \
by Survival, Age and Sex")

points( x=1, y=8, pch=20, col="magenta")
points( x=1, y=4, pch=20, col="green")

title(main=list("Circle area is proportional to \
number of passengers", cex=0.9), line=0.5)

detach(dframe)
\end{verbatim}
 }
%$

\begin{figure}
\includegraphics[width=\textwidth]{Figure4.pdf}
\caption{\label{figure:Figure4}
Balloon plot of all the passengers of Titanic, stratified by survival,
age, sex and class}
\end{figure}

It is now easy to see several facts:
\begin{itemize}
\item A surprisingly large fraction (885/2201) of passengers were crew
      members
\item Most passengers and crew were adult males 
\item Most adult males perished
\item Most women survived, except in $3^{rd}$ class
\item Only 3rd class children perished
\end{itemize}

Perhaps the most striking fact is that survival is lowest among
$3^{rd}$ class passengers for all age and gender groups.  It turns out
that there is a well known reason for this difference in survival.
Passengers in $1^{st}$ and $2^{nd}$ class, as well as crew members,
had better access to the lifeboats.  Since there were too few
lifeboats for the number of passengers and crew, most women and
children among the $1^{st}$ class, $2^{nd}$ class and crew found space
in a lifeboat, while many of the later arriving $3^{rd}$ class women
and children were too late: the lifeboats had already been filled and
had moved away from the quickly sinking ship.

\section*{Discussion}

Our goals in developing the balloonplot were twofold: First, to
improve ability of viewers to quickly perceive trends. Second, to
minimize the need for viewers to learn new idioms.  With these goals
in mind, we have restricted ourselves to simple modifications of the
standard tabular display.  

Other researchers have pursued more general approaches to the visual
display of tabular data. (For a review of that work, see
Hartigan and Kleiner \citep{Hartigan.Kleiner.1981} or 
Friendly \citep{Friendly.1992}.)  One of the most popular methods developed by
these researchers is the \code{mosaic plot} \citep{Snee.1974}.

We have previously experimented with mosaic plots.  Unfortunately,
we found that they do not lend themselves to rapid ascertainment of
trends, particularly by untrained users.  Even trained users find
that they must pay careful attention in order to decode the visual
information presented by the mosaic plot.  In contrast, balloonplots
lend themselves to very quick perception of important trends, even
for users who have never encountered them before.

While there are, of course, tasks for which mosaic plot is
preferable, we feel that the balloonplot serves admirably to allow
high-levels patterns to be quickly perceived by untrained users.

\section*{Conclusion}

Using the well worn Titanic data, we have shown how balloonplots
help to convey important aspects of tabular data, without obscuring
the exact numeric values.  We hope that this new approach to
visualizing tabular data will assist other statisticians in more
effectively understanding and presenting tabular data.


We wish to thank \emph{Ramon Alonso-Allende}
\email{allende@cnb.uam.es} for the discussion on R-help which lead
to the development of \code{balloonplot}, as well as for the code
for displaying the row and column sums.

\address{Gregory R. Warnes, Gregory R. Warnes Statistical Consulting,
         Pittsford, NY, USA\\
\email{greg@warnes.net}\\
       Nitin Jain, Goldman Sachs, USA\\
\email{emailnitinjain@gmail.com}}


\begin{thebibliography}{3}
\expandafter\ifx\csname natexlab\endcsname\relax\def\natexlab#1{#1}\fi
\expandafter\ifx\csname url\endcsname\relax
  \def\url#1{{\tt #1}}\fi

\bibitem[Friendly(1992)]{Friendly.1992}
M.~Friendly.
\newblock Graphical methods for categorical data.
\newblock {\em Proceedings of SAS SUGI 17 Conference}, 1992.

\bibitem[Hartigan and Kleiner(1981)]{Hartigan.Kleiner.1981}
J.~A. Hartigan and B.~Kleiner.
\newblock Mosaics for contingency tables.
\newblock {\em Computer Science and Statistics: Proceedings of the 13th
  Symposium on the Interface. New York: Springer-Verlag}, 1981.

\bibitem[Snee(1974)]{Snee.1974}
R.~D. Snee.
\newblock Graphical display of two-way contingency tables.
\newblock {\em The American Statistician}, 28:\penalty0 9--12, 1974.

\end{thebibliography}

  

\end{article}


\end{document}
